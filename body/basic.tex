\chapter{相关算法基础}
   本文中提出的算法使用了大量最新强化学习算法中的思想,因此有必要介绍强化学习中常用的概念和现有强化学习算法的思想。

    \section{强化学习基础}
    在强化学习中,我们对智能体如何与环境交互感兴趣,这涉及到4个概念:模型、状态、动作和奖励\cite{sutton1988reinforcement}。
    由与智能体和环境相关的可观测量构成了智能体的状态向量$s$,它表示了智能体当前在环境中所处的状态。
    每个状态从状态空间$\mathcal S$中取值,$\mathcal S$包含了所有智能体和环境可以取到的状态。一个智能体可以在某一时刻做出动作$a$并因此转移到下一个状态。
    所有的动作从动作空间$\mathcal A$中取值。如果完全已知一个环境的模型,且已知智能体的当前状态,就可以知道智能体做出任一动作的效果。当智能体做出某一个动作之后,根据当前的任务和环境可以得到一个奖励$r$,它表示智能体从环境中得到的反馈,并从奖励空间$\mathcal R$中取值。
    状态转换、动作和它导致的奖励构成了一个轨迹。如果用$S_t$,$A_t$,$R_t$分别表示在时刻$t$下智能体的状态、动作和奖励,那么轨迹就可以表示为一个序列$S_t,A_t,R_{t+1},S_{t+1},\cdots,S_T$,其中$T$是关心的整个片断结束的时刻。
    策略$\pi(a|s)=\mathbb P[A_t=a|S_t=s]$是智能体在当前状态$S_t=s$时采取动作$A_t=a$的条件概率。
    每个状态都有一个关联的值表示当前状态在特定任务下的价值,用状态价值函数$V_\pi:\mathcal S\to\mathbb R$来表示从一个状态到这个值的映射。
    状态价值函数实际上表示了智能体使用当前策略$\pi$来选择动作,在可能获得奖励的多少。
    类似地,用动作价值函数$Q_\pi:\mathcal S\times\mathcal A\to\mathbb R$表示智能体在给定当前状态和动作后,随后一直使用策略$\pi$来选择动作,在未来获得奖励的多少。
    在在实际算法中,为了防止对价值函数的拟合发散,需要给定一个折扣系数$\gamma\in[0,1]$来减少较远未来获得的奖励对当前价值函数的影响。
    强化学习的最终目标实际上为了最大化累积奖励:
    $$G_t=\sum_{k=0}^\infty \gamma^k R_{t+k+1}$$
    有了累积奖励的定义后,状态价值函数$V_\pi$就可以写成在未来使用策略$\pi$的期望的累积奖励:
    $$V_\pi(s)=\mathbb E_\pi[G_t|S_t=s]$$
    类似地,动作价值函数可以写成:
    $$Q_\pi(s,a)=\mathbb E_\pi[G_t|S_t=s,A_t=a]$$

    \section{TD3算法}
    在通常的演员-评论家模式的强化学习算法中\cite{konda2002actor},演员网络被用于拟合最优的确定性策略$\mu^*:\mathcal S\to\mathcal A$,评论家网络被用于拟合最优的动作价值函数$Q^*:\mathcal S\times \mathcal A\to \mathbb R$。在给定一个迁移$(S_t,A_t,R_t,S_{t+1})$,和现有的演员网络$\mu$和评论家网络$Q$之后,根据如下损失函数对网络参数进行优化:
    $$L_Q = \mathbb E[(R_t + \gamma Q(S_{t+1},\mu(S_{t+1})) - Q(S_t, A_t))^2]$$
    $$L_\mu = \mathbb E[-Q(S_t, \mu(S_t))]$$

    但是由于在上述函数拟合过程中,评论家网络往往倾向于输出更高的动作值或状态价值。
    根据TD3算法\cite{DBLP:journals/corr/abs-1802-09477},可以使用两个评论家网络$Q_1,Q_2:\mathcal S\times\mathcal A\to \mathbb R$。
    这两个评论家网络同时对未来时刻$t+1$时刻的动作和状态进行评估并分别输出价值$Q_1(S_{t+1},A_{t+1})$和$Q_2(S_{t+1},A_{t+1})$。
    取这两个值的最小值作为对未来时刻$t+1$的价值预测,即可得到修正后的评论家网络的损失函数:
    $$L_{Q_1} = \mathbb E[(R_t + \gamma \min\{ Q_1(S_{t+1},\mu(S_{t+1})), Q_2(S_{t+1},\mu(S_{t+1})) \} - Q_1(S_t, A_t))^2]$$
    $$L_{Q_2} = \mathbb E[(R_t + \gamma \min\{ Q_1(S_{t+1},\mu(S_{t+1})), Q_2(S_{t+1},\mu(S_{t+1})) \} - Q_2(S_t, A_t))^2]$$
    在对演员网络进行优化时,只使用评论家网络$Q_1$:
    $$L_\mu = \mathbb E[-Q_1(S_t, \mu(S_t))]$$

    为了防止在训练评论家网络时出现发散或损失不稳定的情况,可以对每个网络引入一个目标网络。
    对每个目标网络权重,不使用损失函数对其进行优化,而是使用原网络和目标网络权重的指数滑动平均进行更新。
    对上述网络$\mu, Q_1, Q_2$,有对应的目标网络$\mu',Q_1',Q_2'$。
    在训练刚开始时,保持原网络和目标网络权重相同。
    在之后的训练过程中,使用上述损失函数对$\mu, Q_1, Q_2$进行更新,而使用如下公式分别对目标网络的权重$W_{\mu'},W_{Q_1'}, W_{Q_2'}$进行更新:
    $$W_{\mu'} = \tau W_\mu + (1-\tau) W_{\mu'}$$
    $$W_{Q_1'} = \tau W_{Q_1} + (1-\tau) W_{Q_1'}$$
    $$W_{Q_2'} = \tau W_{Q_2} + (1-\tau) W_{Q_2'}$$
    其中$\tau$叫做polyak系数,它控制着每次权重更新的多少。

    \section{HER算法}
    事后经验重放(HER)算法是一个用于稀疏奖励的强化学习算法,它主要是为了解决奖励过少导致的训练速度过慢的问题。

    \section{基于局部敏感哈希和计数的探索奖励}

    \section{基于正向动力学预测的探索奖励}
