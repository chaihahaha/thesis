% !Mode:: "TeX:UTF-8"
%%%%%%%%%%%%%%%%%%%%%%%%%%%%%%%%%%%%%%%%%%%%%%%%%%%%%%%%%%%%%%%%%%%%%%%%%%%%%%%%
%          ,
%      /\^/`\
%     | \/   |                CONGRATULATIONS!
%     | |    |             SPRING IS IN THE AIR!
%     \ \    /                                                _ _
%      '\\//'                                               _{ ' }_
%        ||                     hithesis v3                { `.!.` }
%        ||                                                ',_/Y\_,'
%        ||  ,                   dustincys                   {_,_}
%    |\  ||  |\          Email: yanshuoc@gmail.com             |
%    | | ||  | |            https://yanshuo.name             (\|  /)
%    | | || / /                                               \| //
%    \ \||/ /       https://github.com/dustincys/hithesis      |//
%      `\\//`   \\   \./    \\ /     //    \\./   \\   //   \\ |/ /
%     ^^^^^^^^^^^^^^^^^^^^^^^^^^^^^^^^^^^^^^^^^^^^^^^^^^^^^^^^^^^^^^
%%%%%%%%%%%%%%%%%%%%%%%%%%%%%%%%%%%%%%%%%%%%%%%%%%%%%%%%%%%%%%%%%%%%%%%%%%%%%%%%
\documentclass[fontset=fandol,type=master,campus=harbin]{hithesisbook}
% 此处选项中不要有空格
%%%%%%%%%%%%%%%%%%%%%%%%%%%%%%%%%%%%%%%%%%%%%%%%%%%%%%%%%%%%%%%%%%%%%%%%%%%%%%%%
% 必填选项
% type=doctor|master|bachelor|postdoc
%%%%%%%%%%%%%%%%%%%%%%%%%%%%%%%%%%%%%%%%%%%%%%%%%%%%%%%%%%%%%%%%%%%%%%%%%%%%%%%%
% 选填选项(选填选项的缺省值已经尽可能满足了大多数需求,除非明确知道自己有什么
% 需求)
% campus=shenzhen|weihai|harbin
%   含义:校区选项,默认harbin
% glue=true|false
%   含义:由于我工规范中要求字体行距在一个闭区间内,这个选项为true表示tex自
%   动选择,为false表示区间内一个最接近版心要求行数的要求的默认值,缺省值为
%   false。
% tocfour=true|false
%   含义:是否添加第四级目录,只对本科文科个别要求四级目录有效,缺省值为
%   false
% fontset=windows|mac|ubuntu|fandol|adobe
%   含义:设置字体,默认情况会自动识别系统,然后设置字体。后两个是开源字体,自行
%   下载安装后设置使用。windows是中易字库,窝工默认常用字体,绝对没毛病。mac和
%   ubuntu 默认分别是华文和思源字库,理论上用什么字库都行。后两种开源字库的安装
%   方法到谷歌上百度一下什么都有了。Linux非ubuntu发行版、非x86架构机器等如何运行
%   可到github issue上讨论。
% tocblank=true|false
%   含义:目录中第一章之前,是否加一行空白。缺省值为true。
% chapterhang=true|false
%   含义:目录的章标题是否悬挂居中,规范中要求章标题少于15字,所以这个选项
%   有无没什么用,除了特殊需求。缺省值为true。
% fulltime=true|false
%   含义:是否全日制,缺省值为true。非全日制如同等学力等,要在cover中设置类
%   型,封面中不同格式
% subtitle=true|false
%   含义:论文题目是否含有副标题,缺省值为false,如果有要在cover中设置副标
%   题内容,封面中显示。
% newgeometry=one|two|no
%   含义:规范中的自相矛盾之处,版芯是否包含页眉页脚,旧方法是按照包含页眉
%   页脚来设置。该选项是多选选项,如果设置为no,则版新为旧模板的版芯设置方法,
%   如果设置该选项one或two,分别对应两种页眉页码对应版芯线的相对位置。第一种
%   是严格按照规范要求,难看。第二种微调了页眉页码位置,好一点。默认two。
% debug=true|false
%   含义:是否显示版芯框和行号,用来调试。默认否。
% openright=true|false
%   含义:博士论文是否要求章节首页必须在奇数页,此选项不在规范要求中,按个
%   人喜好自行决定。 默认否。注意,窝工的默认情况是打印版博士论文要求右翻页
%   ,电子版要求非右翻页且无空白页。如果想DIY(或身不由己DIY)在什么地方右
%   翻页,将这个选项设置为false,然后在目标位置添加`\cleardoublepage`命令即
%   可。
% library=true|false
%   含义:是否为提交到图书馆的电子版。默认否。注意:如果设置成true,那么
%   openright选项将被强制转换为false。
% capcenterlast=true|false
%   含义:图题、表题最后一行是否居中对齐(我工规范要求居中,但不要求居中对
%   齐),此选项不在规范要求中,按个人喜好自行决定。默认否。
% subcapcenterlast=true|false
%   含义:子图图题最后一行是否居中对齐(我工规范要求居中,但不要求居中对齐
%   ),此选项不在规范要求中,按个人喜好自行决定。默认否。
% absupper=true|false
%   含义:中文目录中的英文摘要在中文目录中的大小写样式歧义,在规范中要求首
%   字母大写,在work样例中是全大写。该选项控制是否全大写。默认否。
% bsmainpagenumberline=true|false
%   含义:由于本科生论文官方模板的页码和页眉格式混乱,提供这个选项自定义设
%   置是否在正文中显示页码横线,默认否。
% bsfrontpagenumberline=true|false
%   含义:由于本科生论文官方模板的页码和页眉格式混乱,提供这个选项自定义设
%   置是否在前文中显示页码横线,默认否。
% bsheadrule=true|false
%   含义:由于本科生论文官方模板的页码和页眉格式混乱,提供这个选项自定义设
%   置是否显示页眉横线,默认显示。
% splitbibitem=true|false
%   含义:参考文献每一个条目内能不能断页,应广大刀客要求添加。默认否。
% newtxmath=true|false
%   含义:数学字体是否使用新罗马。默认是。
% chapterbold=true|false
%   含义:本科生章标题在目录和正文中是否加粗
% engtoc=true|false
%   含义:非博士生需要添加英文目录的,手动添加,如果是博士,此开关无效
% zijv=word|regu
%   含义:字距设置为规范规定33个字还是word中34个字。默认regu。
%%%%%%%%%%%%%%%%%%%%%%%%%%%%%%%%%%%%%%%%%%%%%%%%%%%%%%%%%%%%%%%%%%%%%%%%%%%%%%%%
\usepackage{hithesis}

\graphicspath{{figures/}}

\begin{document}
\frontmatter
% !Mode:: "TeX:UTF-8"

\hitsetup{
  %******************************
  % 注意:
  %   1. 配置里面不要出现空行
  %   2. 不需要的配置信息可以删除
  %******************************
  %
  %=====
  % 秘级
  %=====
  statesecrets={公开},
  natclassifiedindex={TM301.2},
  intclassifiedindex={62-5},
  %
  %=========
  % 中文信息
  %=========
  ctitlecover={基于虚拟物理仿真思考的开放任务求解},%放在封面中使用,自由断行
  ctitle={基于虚拟物理仿真思考的开放任务求解},%放在原创性声明中使用
  cxueke={工程},
  csubject={软件工程},
  caffil={计算学部},
  cauthor={柴士童},
  csupervisor={范晓鹏},
  % 日期自动使用当前时间,若需指定按如下方式修改:
  cdate={2020年12月},
  cstudentid={18S103239},
  cstudenttype={全日制工程硕士}, %非全日制教育申请学位者
  %cnumber={no9527}, %编号
  %cpositionname={哈铁西站}, %博士后站名称
  %cfinishdate={20XX年X月---20XX年X月}, %到站日期
  %csubmitdate={20XX年X月}, %出站日期
  %cstartdate={3050年9月10日}, %到站日期
  %cenddate={3090年10月10日}, %出站日期
  %(同等学力人员)、(工程硕士)、(工商管理硕士)、
  %(高级管理人员工商管理硕士)、(公共管理硕士)、(中职教师)、(高校教师)等
  %
  %
  %=========
  % 英文信息
  %=========
  etitle={Physical Simulation and Reasoning based Task-Agnostic learning},
  exueke={Technology},
  esubject={Software Engineering},
  eaffil={\emultiline[t]{Faculty of Computing}},
  eauthor={CHAI Shitong},
  esupervisor={Professor FAN Xiaopeng},
  %eassosupervisor={XXX},
  % 日期自动生成,若需指定按如下方式修改:
  edate={December, 2020},
  estudenttype={Master of Engineering},
  %
  % 关键词用“英文逗号”分割
  ckeywords={元学习, 强化学习, 机器人学, 物理仿真, 深度学习},
  ekeywords={Meta Learning, Reinforcement Learning, Robotics, Physics Simulation, Deep Learning},
}

\begin{cabstract}
本文在Pyrobolearn机器人仿真环境中设计了一个要求机械臂的末端执行器到达指定物体附近的开放任务,并设计了新的强化学习算法用于训练智能体在未获得任务相关的奖励之前在未知环境中学习到可泛化到该开放任务的策略。

在TD3算法和HER算法的基础上,本文引入了基于局部敏感哈希的基于计数的奖励和基于正向动力学预测的奖励用于鼓励智能体探索未知环境。并提出了高斯混合噪声层、和基于物理仿真引擎的奖励。


本文提出的斯混合噪声层被用于提供自适应的策略噪声。本文提出的基于物理仿真引擎仿真时间的奖励被用于在未知任务的环境中鼓励智能体学习有效可泛化的策略,并在获得稀疏奖励的开放任务后快速适应到新的策略。

\end{cabstract}

\begin{eabstract}
A task which requires the end effector of a manipulator to reach a specific body is designed with the help of the robot learning framework Pyrobolearn. New reinforcement learning algorithms are proposed to train an agent to learn in a task-agnostic environment without reward related to the task, where the learnt policy of the agent should be generalizable to the proposed task.

Based on TD3 and HER algorithms, a locality sensitive hashing based reward and a forward dynamics prediction model based reward are introduced to encourage the agent to explore in the unknown environment. A mixed gaussian noise layer is proposed to provide a adaptive policy noise. A reward based on physics simulation time is proposed to encourage the agent to learn a generalizable policy in a task agnostic environment, and adapt to a new policy after given a task setting with sparse reward.
\end{eabstract}
 % 封面
\makecover
\begin{denotation}
\begin{table}[h]%此处最好是h
\caption{国际单位制中具有专门名称的导出单位}
\vspace{0.5em}\centering\wuhao
\begin{tabular}{ccccc}
\toprule[1.5pt]
量的名称&单位名称&单位符号&其它表示实例\\
\midrule[1pt]
频率&赫[兹]&Hz&s-1\\
\bottomrule[1.5pt]
\end{tabular}
\end{table}
\end{denotation}
%物理量名称表,符合规范为主,有要求添加
\tableofcontents %目录
\mainmatter
% !Mode:: "TeX:UTF-8"

\chapter[哈尔滨工业大学研究生学位论文撰写规范]{哈尔滨工业大学研究生学
  位论文\protect\\撰写规范}[Harbin Institute of Technology Postgraduate Dissertation Writing Specifications]

研究生学位论文是研究生科学研究工作的全面总结,是描述其研究成果、代表其研究水平的
重要学术文献资料,是申请和授予相应学位的基本依据。学位论文撰写是研究生培养过程的
基本训练之一,必须按照确定的规范认真执行。研究生应严肃认真地撰写学位论文,指导教
师应加强指导,严格把关。

学位论文撰写应实事求是,杜绝造假和抄袭等行为;应符合国家及各专业部门制定的有关标
准,符合汉语语法规范。硕士和博士学位论文,除在字数、理论研究的深度及创造性成果等
方面的要求不同外,撰写规范要求基本一致。人文与社会科学、管理学科可在本撰写规范的
基础上补充制定专业的学术规范。

\section{内容要求}[Content specification]
\subsection{题目}[Title]

题目应以简明的词语,恰当、准确、科学地反映论文最重要的特定内容(一般不超过25字),
应中英文对照。题目通常由名词性短语构成,不能含有标点符号;应尽量避免使用不常用的
缩略词、首字母缩写字、字符、代号和公式等。

如题目内容层次很多,难以简化时,可采用题目和副题目相结合的方法。题目与副题目字数
之和不应超过35字,中文的题目与副题目之间用破折号相连,英文则用冒号相连。副题目起
补充、阐明题目的作用。题目和副题目在整篇学位论文中的不同地方出现时,应保持一致。

\subsection{摘要与关键词}[Abstraction and key words]
\subsubsection{摘要}[Abstraction]

摘要是论文内容的高度概括,应具有独立性和自含性,即不阅读论文的全文,就能通过摘要
了解整个论文的必要信息。摘要应包括本论文研究的目的、理论与实际意义、主要研究内容、
研究方法等,重点突出研究成果和结论。

摘要的内容要完整、客观、准确,应做到不遗漏、不拔高、不添加。摘要应按层次逐段简要
写出,避免将摘要写成目录式的内容介绍。摘要在叙述研究内容、研究方法和主要结论时,
除作者的价值和经验判断可以使用第一人称外,一般使用第三人称,采用“分析了……原因”、
“认为……”、“对……进行了探讨”等记述方法进行描述。避免主观性的评价意见,避免
对背景、目的、意义、概念和一般性(常识性)理论叙述过多。

摘要需采用规范的名词术语(包括地名、机构名和人名)。对个别新术语或无中文译文的术
语,可用外文或在中文译文后加括号注明外文。摘要中不宜使用公式、化学结构式、图表、
非常用的缩写词和非公知公用的符号与术语,不标注引用文献编号。

博士学位论文摘要应包括以下几个方面的内容:

(1)论文的研究背景及目的。简洁准确地交代论文的研究背景与意义、相关领域的研究现
状、论文所针对的关键科学问题,使读者把握论文选题的必要性和重要性。此部分介绍不宜
写得过多,一般不多于400字。

(2)论文的主要研究内容。介绍论文所要解决核心问题开展的主要研究工作以及研究方法
或研究手段,使读者可以了解论文的研究思路、研究方案、研究方法或手段的合理性与先进
性。

(3)论文的主要创新成果。简要阐述论文的新思想、新观点、新技术、新方法、新结论等
主要信息,使读者可以了解论文的创新性。

(4)论文成果的理论和实际意义。客观、简要地介绍论文成果的理论和实际意义,使读者
可以快速获得论文的学术价值。

\subsubsection{关键词}[Keywords]
关键词是供检索用的主题词条。关键词应集中体现论文特色,反映研究成果的内涵,具有语
义性,在论文中有明确的出处,并应尽量采用《汉语主题词表》或各专业主题词表提供的规
范词,应列取3$\sim$6个关键词,按词条的外延层次从大到小排列。

\subsection{目录}[Content]

论文中各章节的顺序排列表,包括论文中全部章、节、条三级标题及其页码。

\subsection{论文正文}[Main body]

论文正文包括绪论、论文主体及结论等部分。

\subsubsection{绪论}
绪论一般作为第1章。绪论应包括:本研究课题的来源、背景及其理论意义与实际意义;国
内外与课题相关研究领域的研究进展及成果、存在的不足或有待深入研究的问题,归纳出将
要开展研究的理论分析框架、研究内容、研究程序和方法。

绪论部分要注意对论文所引用国内外文献的准确标注。绪论的主要研究内容的撰写宜使用将
来时态,切忌将论文目录直接作为研究内容。

\subsubsection{论文主体}
论文主体是学位论文的主要部分,应该结构严谨,层次清晰,重点突出,文字简练、通顺。
论文各章之间应该前后关联,构成一个有机的整体。论文给出的数据必须真实可靠,推理正
确,结论明确,无概念性和科学性错误。对于科学实验、计算机仿真的条件、实验过程、仿
真过程等需加以叙述,避免直接给出结果、曲线和结论。引用他人研究成果或采用他人成说
时,应注明出处,不得将其与本人提出的理论分析混淆在一起。

论文主体各章后应有一节“本章小结”,实验方法或材料等章节可不写“本章小结”。各章
小结是对各章研究内容、方法与成果的简洁准确的总结与概括,也是论文最后结论的依据。

\subsubsection{结论}
结论作为学位论文正文的组成部分,单独排写,不加章标题序号,不标注




% Local Variables:
% TeX-master: "../main"
% TeX-engine: xetex
% End:
\chapter{绪论}

    \section{选题背景及研究意义}
    
        \subsection{选题背景}
        一直以来,强化学习和机器人学都是人工智能研究中的热门领域。在2008年,Deepmind团队基于深度强化学习研发的围棋人工智能系统AlphaGo Zero在零知识自我对弈的情况下在几天之内超越了旧的系统AlphaGo,而AlphaGo曾击败了围棋领域中世界公认的专家柯洁等人\cite{silver2018general}。这项研究使越来越多的人们开始关注人工智能领域,并使得强化学习成为研究热点。事实上,强化学习已经成为最热门的研究领域之一,并在自动控制、运筹学、机器人学、游戏智能体和无人驾驶等领域中获得了广泛的应用\cite{dosovitskiy2017carla}。在这些领域中,机器人学是和前沿强化学习算法关系最密切的领域之一。传统的机器人如机械臂、四足机器人等,可以用强化学习训练得到的智能体进行控制,并在与环境交互过程中根据环境反馈使策略得到进一步的优化。

        然而,由于机器人设计和制造的成本较高,通用的多关节机器人通常非常昂贵。而且机器人通常容易在强化学习中的各种随机探索中受到损坏,并导致控制系统和智能体策略在训练过程中出现错误。因此在实际的机器人上进行所有强化学习训练是不现实的\cite{toussaint2018differentiable, todorov2012mujoco}。为了避免这个问题,可以使用物理仿真引擎对机器人和环境进行建模,并在实际测试智能体之前先在仿真环境中对智能体进行训练。幸运的是,随着强化学习仿真需求的增加,越来越多的针对机器人的仿真环境开始出现,在这些仿真环境中,可以像在真实环境中一样控制机器人的关节、调节各种参数,或获得传感器数据等等,并可以做到在真实环境中难以做到的设定复杂的稀疏奖励、获取碰撞次数、和改变环境的物理参数等操作\cite{savva2019habitat}。

        虽然已经有大量的软件系统可以用于在一个环境建模完全精确的情形下解决一个良定义的任务\cite{toussaint2018differentiable},如何让智能体在面对未知的新环境和未知的新任务后能够有效泛化之前学习到的策略仍然是一个未完全解决的问题。人类可以在陌生的环境中用很少次数的探索自然地掌握大量有效信息,还可以利用已有经验对大量物体进行分类、提高对物理实体运动的预测能力,或创新性地设计工具解决问题。由于人们对大脑的工作原理仍然知之甚少,这个过程通常很难被数值化为一个单一的奖励函数或强化学习算法。

        本课题致力于解决上述问题,即设计算法从而可以训练出能在任务奖励未知的环境中进行探索,获取环境信息,学习基本策略,并在任务确定后快速调节旧策略以适应新任务的智能体。为了实现这个目标,需要利用现有的开源物理仿真引擎、前沿的强化学习算法和具有强大函数拟合能力的深度神经网络来设计新算法。
        
        \subsection{研究意义}

        机器人控制对于工业制造有着重要的意义,在工厂流水线上,机械臂常常被设计为只能完成单个简单任务,或需要工人远程控制。虽然它们已经极大地提高了生产效率,减少了对工人们生命财产安全的威胁,但是机械臂由于价格昂贵,仅仅用于单一任务会造成极大的资源浪费。

        本课题提出的算法有希望训练出可以在对未知任务奖励的环境进行充分探索之后,快速适应多种不同任务的机器人智能体,从而扩大现有强化学习算法的适用范围,解决更复杂的控制任务,增强机器人智能体泛化策略的能力。

        此外,本课题还可以加深对现有强化学习算法在机器人控制中应用价值的理解,可以通过机器人仿真和控制帮助提前发现在应用算法到实际机器人控制时可能出现的问题,可以通过设计和调整深度神经网络进一步了解不同结构的神经网络对智能体性能的影响。
    
    \section{国内外研究进展}

    国内外已经有了很多关于让机器人智能体使用工具和泛化已学习的策略到新任务的研究。其中关注工具使用、无监督兴趣导向的探索、模仿学习和自驱动的主动学习与本课题有关。
        \subsection{基于引导视觉预见的工具使用}
        引导视觉预见\cite{xie2019improvisation}可以使机械臂智能体从人类演示中学习并泛化学习到的能力以在不同的环境中使用工具。这种方法包含动作提案模型和预测模型。其中动作提案模型使用演示动作数据来训练一个自回归的长短时记忆网络模型来根据图像传感器拍摄到的图像数据生成动作。预测模型是基于卷积长短时记忆网络的\cite{shi2015convolutional}。预测模型被用于对物理实体的运动进行物理预测,并可以被用于筛选不能完成指定目标的动作序列。训练过程分为基于演示的模仿学习和利用握爪反射的随机自动训练。在测试过程中,指定的目标被定义为像素移动,预测模型输出的像素位置和真实像素位置的距离被用于评估动作好坏。在指定任务后,动作序列从动作提案模型中采样,并使用交叉熵方法结合预测模型进行优化。实验表明,此种方法可以比单纯模仿学习在新环境中获得更好的泛化性能。

        \subsection{无监督兴趣导向探索}
        引导视觉预见需要有人类使用工具的演示数据才能正常工作,而对于更一般的环境,人类的演示数据可能是无法获得的,此时智能体应当可以在没有指定任务的情况下在环境中探索。为了解决这个问题,一个兴趣导向的探索方法被提出了\cite{laversanne-finot2018curiosity}。这个方法结合了解纠缠目标空间的变分自编码机(VAE)和兴趣导向的IMGEP(Intrinsically Motivated Goal Exploration Process)方法\cite{DBLP:journals/corr/abs-1708-02190}。在目标空间被给定后,IMGEP框架下的智能体会倾向于选择更有可能增加竞争力的目标。因为不像通常的强化学习算法一样在给定单一目标后做训练,而是无监督地选择目标进行训练,因此这是一个元策略算法。在探索过程中,智能体会学习到被$\beta$-VAE解纠缠后的观测,因此智能体可以分离地探索不同的物体。在实验中,此种方法获得了比单纯IMGEP方法更大的探索率。

        \subsection{模仿学习}
        在需要使用工具求解的开放任务中,往往有着稀疏奖励,随机探索很难刚好完成一个完整的动作序列,最终成功地使用工具完成任务,并获得奖励。
        模仿学习通过人类的专家策略,可以极大地加速这种随机探索过程。通过人类提供的演示数据,智能体应当能够学习到更好的探索策略,增大获得奖励的概率。
        相关研究表明,使用常规的强化学习算法和运动剪辑数据集,智能体可以学会组合学到的不同的技能,并用于求解多种任务\cite{peng2018deepmimic}。不仅如此,智能体也可以从视频中学习到一些技能\cite{peng2019sfv}。这意味着现有模仿学习方法可以利用网络中大量的视频数据进行学习。

        \subsection{自驱动的主动学习}
        在开放任务求解中,主动学习技术也可以被使用。为了主动学习预测物理环境的能力,可以使用带策略的循环Q网络来减少未知物理性质的熵\cite{li2019active}。
        在机器人学中,逆动力学模型对于稳健控制非常重要。一种主动学习方法使用竞争力来选择目标并在高维连续空间中学习各种技能,并用于机器人控制\cite{baranes2013active}。仿真结果表明这种方法能够帮助机器人智能体探索到随机策略难以探索到的区域。

    \section{研究内容和方法}

        \subsection{研究内容}

        \subsection{研究方法}


\backmatter
% !Mode:: "TeX:UTF-8" 
\begin{conclusions}

    本文结合了TD3算法、HER算法和基于LSH和计数的奖励、基于正向动力学预测的奖励,并提出了混合高斯噪声层和将物理仿真时间作为于未知任务的探索奖励。本文中的算法可以很好地用于训练可适应到给定开放任务的智能体,表现出与现有探索奖励相当的性能。
\end{conclusions}
   % 结论
\bibliographystyle{hithesis} %如果没有参考文献时候
\bibliography{reference}
%%%%%%%%%%%%%%%%%%%%%%%%%%%%%%%%%%%%%%%%%%%%%%%%%%%%%%%%%%%%%%%%%%%%%%%%%%%%%%%% 
%-- 注意:以下本硕博、博后书序不一致 --%
%%%%%%%%%%%%%%%%%%%%%%%%%%%%%%%%%%%%%%%%%%%%%%%%%%%%%%%%%%%%%%%%%%%%%%%%%%%%%%%% 
% 硕博书序
%%%%%%%%%%%%%%%%%%%%%%%%%%%%%%%%%%%%%%%%%%%%%%%%%%%%%%%%%%%%%%%%%%%%%%%%%%%%%%%% 
\begin{appendix}%附录
% -*-coding: utf-8 -*-
%%%%%%%%%%%%%%%%%%%%%%%%%%%%%%%%%%%%%%%%%%%%%%%%%%%%%%%%%
\chapter{算法代码}

%%%%%%%%%%%%%%%%%%%%%%%%%%%%%%%%%%%%%%%%%%%%%%%%%%%%%%%%%
\section{训练代码}

\section{测试代码}

\end{appendix}
% !Mode:: "TeX:UTF-8" 
\begin{publication}
\noindent\textbf{发表的相关论文}
\begin{publist}
\item	XXX,XXX. Static Oxidation Model of Al-Mg/C Dissipation Thermal Protection Materials[J]. Rare Metal Materials and Engineering, 2010, 39(Suppl. 1): 520-524.(SCI~收录,IDS号为~669JS,IF=0.16)
\item XXX,XXX. 精密超声振动切削单晶铜的计算机仿真研究[J]. 系统仿真学报,2007,19(4):738-741,753.(EI~收录号:20071310514841)
\item XXX,XXX. 局部多孔质气体静压轴向轴承静态特性的数值求解[J]. 摩擦学学报,2007(1):68-72.(EI~收录号:20071510544816)
\item XXX,XXX. 硬脆光学晶体材料超精密切削理论研究综述[J]. 机械工程学报,2003,39(8):15-22.(EI~收录号:2004088028875)
\item XXX,XXX. 基于遗传算法的超精密切削加工表面粗糙度预测模型的参数辨识以及切削参数优化[J]. 机械工程学报,2005,41(11):158-162.(EI~收录号:2006039650087)
\item XXX,XXX. Discrete Sliding Mode Cintrok with Fuzzy Adaptive Reaching Law on 6-PEES Parallel Robot[C]. Intelligent System Design and Applications, Jinan, 2006: 649-652.(EI~收录号:20073210746529)
\end{publist}

\noindent\textbf{(二)申请及已获得的专利(无专利时此项不必列出)}
\begin{publist}
\item XXX,XXX. 一种温热外敷药制备方案:中国,88105607.3[P]. 1989-07-26.
\end{publist}

\noindent\textbf{(三)参与的科研项目及获奖情况}
\begin{publist}
\item	XXX,XXX. XX~气体静压轴承技术研究, XX~省自然科学基金项目.课题编号:XXXX.
\item XXX,XXX. XX~静载下预应力混凝土房屋结构设计统一理论. 黑江省科学技术二等奖, 2007.
\end{publist}
%\vfill
%\hangafter=1\hangindent=2em\noindent
%\setlength{\parindent}{2em}
\end{publication}
    % 所发文章
\begin{ceindex}
  %如果想要手动加索引,注释掉以下这一样,用wordlist环境
\printsubindex*
\end{ceindex}
    % 索引, 根据自己的情况添加或者不添加,选择自动添加或者手工添加。
\authorization %授权
%\authorization[scan.pdf] %添加扫描页的命令,与上互斥
% !Mode:: "TeX:UTF-8"
\begin{acknowledgements}
衷心感谢导师~范晓鹏~教授对本人的精心指导。他的言传身教将使我终生受益。

感谢哈工大\LaTeX\ 论文模板\hithesis\ 。

\end{acknowledgements}
 %致谢
\include{back/resume}          % 博士学位论文有个人简介
%%%%%%%%%%%%%%%%%%%%%%%%%%%%%%%%%%%%%%%%%%%%%%%%%%%%%%%%%%%%%%%%%%%%%%%%%%%%%%%% 
% 本科书序为:
%%%%%%%%%%%%%%%%%%%%%%%%%%%%%%%%%%%%%%%%%%%%%%%%%%%%%%%%%%%%%%%%%%%%%%%%%%%%%%%% 
% \authorization %授权
% % \authorization[scan.pdf] %添加扫描页的命令,与上互斥
% % !Mode:: "TeX:UTF-8"
\begin{acknowledgements}
衷心感谢导师~范晓鹏~教授对本人的精心指导。他的言传身教将使我终生受益。

感谢哈工大\LaTeX\ 论文模板\hithesis\ 。

\end{acknowledgements}
 %致谢
% \begin{appendix}%附录
% \chapter{外文资料原文}
\label{cha:engorg}

\title{The title of the English paper}

\textbf{Abstract:} As one of the most widely used techniques in operations
research, \emph{ mathematical programming} is defined as a means of maximizing a
quantity known as \emph{bjective function}, subject to a set of constraints
represented by equations and inequalities. Some known subtopics of mathematical
programming are linear programming, nonlinear programming, multiobjective
programming, goal programming, dynamic programming, and multilevel
programming$^{[1]}$.

It is impossible to cover in a single chapter every concept of mathematical
programming. This chapter introduces only the basic concepts and techniques of
mathematical programming such that readers gain an understanding of them
throughout the book$^{[2,3]}$.


\section{Single-Objective Programming}
The general form of single-objective programming (SOP) is written
as follows,
\begin{equation}\tag*{(123)} % 如果附录中的公式不想让它出现在公式索引中,那就请
                             % 用 \tag*{xxxx}
\left\{\begin{array}{l}
\max \,\,f(x)\\[0.1 cm]
\mbox{subject to:} \\ [0.1 cm]
\qquad g_j(x)\le 0,\quad j=1,2,\cdots,p
\end{array}\right.
\end{equation}
which maximizes a real-valued function $f$ of
$x=(x_1,x_2,\cdots,x_n)$ subject to a set of constraints.

\newtheorem{mpdef}{Definition}[chapter]
\begin{mpdef}
In SOP, we call $x$ a decision vector, and
$x_1,x_2,\cdots,x_n$ decision variables. The function
$f$ is called the objective function. The set
\begin{equation}\tag*{(456)} % 这里同理,其它不再一一指定。
S=\left\{x\in\Re^n\bigm|g_j(x)\le 0,\,j=1,2,\cdots,p\right\}
\end{equation}
is called the feasible set. An element $x$ in $S$ is called a
feasible solution.
\end{mpdef}

\newtheorem{mpdefop}[mpdef]{Definition}
\begin{mpdefop}
A feasible solution $x^*$ is called the optimal
solution of SOP if and only if
\begin{equation}
f(x^*)\ge f(x)
\end{equation}
for any feasible solution $x$.
\end{mpdefop}

One of the outstanding contributions to mathematical programming was known as
the Kuhn-Tucker conditions\ref{eq:ktc}. In order to introduce them, let us give
some definitions. An inequality constraint $g_j(x)\le 0$ is said to be active at
a point $x^*$ if $g_j(x^*)=0$. A point $x^*$ satisfying $g_j(x^*)\le 0$ is said
to be regular if the gradient vectors $\nabla g_j(x)$ of all active constraints
are linearly independent.

Let $x^*$ be a regular point of the constraints of SOP and assume that all the
functions $f(x)$ and $g_j(x),j=1,2,\cdots,p$ are differentiable. If $x^*$ is a
local optimal solution, then there exist Lagrange multipliers
$\lambda_j,j=1,2,\cdots,p$ such that the following Kuhn-Tucker conditions hold,
\begin{equation}
\label{eq:ktc}
\left\{\begin{array}{l}
    \nabla f(x^*)-\sum\limits_{j=1}^p\lambda_j\nabla g_j(x^*)=0\\[0.3cm]
    \lambda_jg_j(x^*)=0,\quad j=1,2,\cdots,p\\[0.2cm]
    \lambda_j\ge 0,\quad j=1,2,\cdots,p.
\end{array}\right.
\end{equation}
If all the functions $f(x)$ and $g_j(x),j=1,2,\cdots,p$ are convex and
differentiable, and the point $x^*$ satisfies the Kuhn-Tucker conditions
(\ref{eq:ktc}), then it has been proved that the point $x^*$ is a global optimal
solution of SOP.

\subsection{Linear Programming}
\label{sec:lp}

If the functions $f(x),g_j(x),j=1,2,\cdots,p$ are all linear, then SOP is called
a {\em linear programming}.

The feasible set of linear is always convex. A point $x$ is called an extreme
point of convex set $S$ if $x\in S$ and $x$ cannot be expressed as a convex
combination of two points in $S$. It has been shown that the optimal solution to
linear programming corresponds to an extreme point of its feasible set provided
that the feasible set $S$ is bounded. This fact is the basis of the {\em simplex
  algorithm} which was developed by Dantzig as a very efficient method for
solving linear programming.
\begin{table}[ht]
\centering
  \centering
  \caption*{Table~1\hskip1em This is an example for manually numbered table, which
    would not appear in the list of tables}
  \label{tab:badtabular2}
  \begin{tabular}[c]{|m{1.5cm}|c|c|c|c|c|c|}\hline
    \multicolumn{2}{|c|}{Network Topology} & \# of nodes &
    \multicolumn{3}{c|}{\# of clients} & Server \\\hline
    GT-ITM & Waxman Transit-Stub & 600 &
    \multirow{2}{2em}{2\%}&
    \multirow{2}{2em}{10\%}&
    \multirow{2}{2em}{50\%}&
    \multirow{2}{1.2in}{Max. Connectivity}\\\cline{1-3}
    \multicolumn{2}{|c|}{Inet-2.1} & 6000 & & & &\\\hline
    & \multicolumn{2}{c|}{ABCDEF} &\multicolumn{4}{c|}{} \\\hline
\end{tabular}
\end{table}

Roughly speaking, the simplex algorithm examines only the extreme points of the
feasible set, rather than all feasible points. At first, the simplex algorithm
selects an extreme point as the initial point. The successive extreme point is
selected so as to improve the objective function value. The procedure is
repeated until no improvement in objective function value can be made. The last
extreme point is the optimal solution.

\subsection{Nonlinear Programming}

If at least one of the functions $f(x),g_j(x),j=1,2,\cdots,p$ is nonlinear, then
SOP is called a {\em nonlinear programming}.

A large number of classical optimization methods have been developed to treat
special-structural nonlinear programming based on the mathematical theory
concerned with analyzing the structure of problems.

Now we consider a nonlinear programming which is confronted solely with
maximizing a real-valued function with domain $\Re^n$.  Whether derivatives are
available or not, the usual strategy is first to select a point in $\Re^n$ which
is thought to be the most likely place where the maximum exists. If there is no
information available on which to base such a selection, a point is chosen at
random. From this first point an attempt is made to construct a sequence of
points, each of which yields an improved objective function value over its
predecessor. The next point to be added to the sequence is chosen by analyzing
the behavior of the function at the previous points. This construction continues
until some termination criterion is met. Methods based upon this strategy are
called {\em ascent methods}, which can be classified as {\em direct methods},
{\em gradient methods}, and {\em Hessian methods} according to the information
about the behavior of objective function $f$. Direct methods require only that
the function can be evaluated at each point. Gradient methods require the
evaluation of first derivatives of $f$. Hessian methods require the evaluation
of second derivatives. In fact, there is no superior method for all
problems. The efficiency of a method is very much dependent upon the objective
function.

\subsection{Integer Programming}

{\em Integer programming} is a special mathematical programming in which all of
the variables are assumed to be only integer values. When there are not only
integer variables but also conventional continuous variables, we call it {\em
  mixed integer programming}. If all the variables are assumed either 0 or 1,
then the problem is termed a {\em zero-one programming}. Although integer
programming can be solved by an {\em exhaustive enumeration} theoretically, it
is impractical to solve realistically sized integer programming problems. The
most successful algorithm so far found to solve integer programming is called
the {\em branch-and-bound enumeration} developed by Balas (1965) and Dakin
(1965). The other technique to integer programming is the {\em cutting plane
  method} developed by Gomory (1959).

\hfill\textit{Uncertain Programming\/}\quad(\textsl{BaoDing Liu, 2006.2})

\section*{References}
\noindent{\itshape NOTE: These references are only for demonstration. They are
  not real citations in the original text.}

\begin{translationbib}
\item Donald E. Knuth. The \TeX book. Addison-Wesley, 1984. ISBN: 0-201-13448-9
\item Paul W. Abrahams, Karl Berry and Kathryn A. Hargreaves. \TeX\ for the
  Impatient. Addison-Wesley, 1990. ISBN: 0-201-51375-7
\item David Salomon. The advanced \TeX book.  New York : Springer, 1995. ISBN:0-387-94556-3
\end{translationbib}

\chapter{外文资料的调研阅读报告或书面翻译}

\title{英文资料的中文标题}

{\heiti 摘要:} 本章为外文资料翻译内容。如果有摘要可以直接写上来,这部分好像没有
明确的规定。

\section{单目标规划}
北冥有鱼,其名为鲲。鲲之大,不知其几千里也。化而为鸟,其名为鹏。鹏之背,不知其几
千里也。怒而飞,其翼若垂天之云。是鸟也,海运则将徙于南冥。南冥者,天池也。
\begin{equation}\tag*{(123)}
 p(y|\mathbf{x}) = \frac{p(\mathbf{x},y)}{p(\mathbf{x})}=
\frac{p(\mathbf{x}|y)p(y)}{p(\mathbf{x})}
\end{equation}

吾生也有涯,而知也无涯。以有涯随无涯,殆已!已而为知者,殆而已矣!为善无近名,为
恶无近刑,缘督以为经,可以保身,可以全生,可以养亲,可以尽年。

\subsection{线性规划}
庖丁为文惠君解牛,手之所触,肩之所倚,足之所履,膝之所倚,砉然响然,奏刀騞然,莫
不中音,合于桑林之舞,乃中经首之会。
\begin{table}[ht]
\centering
  \centering
  \caption*{表~1\hskip1em 这是手动编号但不出现在索引中的一个表格例子}
  \label{tab:badtabular3}
  \begin{tabular}[c]{|m{1.5cm}|c|c|c|c|c|c|}\hline
    \multicolumn{2}{|c|}{Network Topology} & \# of nodes &
    \multicolumn{3}{c|}{\# of clients} & Server \\\hline
    GT-ITM & Waxman Transit-Stub & 600 &
    \multirow{2}{2em}{2\%}&
    \multirow{2}{2em}{10\%}&
    \multirow{2}{2em}{50\%}&
    \multirow{2}{1.2in}{Max. Connectivity}\\\cline{1-3}
    \multicolumn{2}{|c|}{Inet-2.1} & 6000 & & & &\\\hline
    & \multicolumn{2}{c|}{ABCDEF} &\multicolumn{4}{c|}{} \\\hline
\end{tabular}
\end{table}

文惠君曰:“嘻,善哉!技盖至此乎?”庖丁释刀对曰:“臣之所好者道也,进乎技矣。始臣之
解牛之时,所见无非全牛者;三年之后,未尝见全牛也;方今之时,臣以神遇而不以目视,
官知止而神欲行。依乎天理,批大郤,导大窾,因其固然。技经肯綮之未尝,而况大坬乎!
良庖岁更刀,割也;族庖月更刀,折也;今臣之刀十九年矣,所解数千牛矣,而刀刃若新发
于硎。彼节者有间而刀刃者无厚,以无厚入有间,恢恢乎其于游刃必有余地矣。是以十九年
而刀刃若新发于硎。虽然,每至于族,吾见其难为,怵然为戒,视为止,行为迟,动刀甚微,
謋然已解,如土委地。提刀而立,为之而四顾,为之踌躇满志,善刀而藏之。”

文惠君曰:“善哉!吾闻庖丁之言,得养生焉。”


\subsection{非线性规划}
孔子与柳下季为友,柳下季之弟名曰盗跖。盗跖从卒九千人,横行天下,侵暴诸侯。穴室枢
户,驱人牛马,取人妇女。贪得忘亲,不顾父母兄弟,不祭先祖。所过之邑,大国守城,小
国入保,万民苦之。孔子谓柳下季曰:“夫为人父者,必能诏其子;为人兄者,必能教其弟。
若父不能诏其子,兄不能教其弟,则无贵父子兄弟之亲矣。今先生,世之才士也,弟为盗
跖,为天下害,而弗能教也,丘窃为先生羞之。丘请为先生往说之。”

柳下季曰:“先生言为人父者必能诏其子,为人兄者必能教其弟,若子不听父之诏,弟不受
兄之教,虽今先生之辩,将奈之何哉?且跖之为人也,心如涌泉,意如飘风,强足以距敌,
辩足以饰非。顺其心则喜,逆其心则怒,易辱人以言。先生必无往。”

孔子不听,颜回为驭,子贡为右,往见盗跖。

\subsection{整数规划}
盗跖乃方休卒徒大山之阳,脍人肝而餔之。孔子下车而前,见谒者曰:“鲁人孔丘,闻将军
高义,敬再拜谒者。”谒者入通。盗跖闻之大怒,目如明星,发上指冠,曰:“此夫鲁国之
巧伪人孔丘非邪?为我告之:尔作言造语,妄称文、武,冠枝木之冠,带死牛之胁,多辞缪
说,不耕而食,不织而衣,摇唇鼓舌,擅生是非,以迷天下之主,使天下学士不反其本,妄
作孝弟,而侥幸于封侯富贵者也。子之罪大极重,疾走归!不然,我将以子肝益昼餔之膳。”


\chapter{其它附录}
前面两个附录主要是给本科生做例子。其它附录的内容可以放到这里,当然如果你愿意,可
以把这部分也放到独立的文件中,然后将其到主文件中。
%本科生翻译论文
% \end{appendix}
%%%%%%%%%%%%%%%%%%%%%%%%%%%%%%%%%%%%%%%%%%%%%%%%%%%%%%%%%%%%%%%%%%%%%%%%%%%%%%%% 
% 博后书序
%%%%%%%%%%%%%%%%%%%%%%%%%%%%%%%%%%%%%%%%%%%%%%%%%%%%%%%%%%%%%%%%%%%%%%%%%%%%%%%% 
% % !Mode:: "TeX:UTF-8"
\begin{acknowledgements}
衷心感谢导师~范晓鹏~教授对本人的精心指导。他的言传身教将使我终生受益。

感谢哈工大\LaTeX\ 论文模板\hithesis\ 。

\end{acknowledgements}
 %致谢
% % !Mode:: "TeX:UTF-8" 

\begin{doctorpublication}
\noindent\textbf{(一)发表的学术论文}
\begin{publist}
\item	XXX,XXX. Static Oxidation Model of Al-Mg/C Dissipation Thermal Protection Materials[J]. Rare Metal Materials and Engineering, 2010, 39(Suppl. 1): 520-524.(SCI~收录,IDS号为~669JS,IF=0.16)
\item XXX,XXX. 精密超声振动切削单晶铜的计算机仿真研究[J]. 系统仿真学报,2007,19(4):738-741,753.(EI~收录号:20071310514841)
\item XXX,XXX. 局部多孔质气体静压轴向轴承静态特性的数值求解[J]. 摩擦学学报,2007(1):68-72.(EI~收录号:20071510544816)
\item XXX,XXX. 硬脆光学晶体材料超精密切削理论研究综述[J]. 机械工程学报,2003,39(8):15-22.(EI~收录号:2004088028875)
\item XXX,XXX. 基于遗传算法的超精密切削加工表面粗糙度预测模型的参数辨识以及切削参数优化[J]. 机械工程学报,2005,41(11):158-162.(EI~收录号:2006039650087)
\item XXX,XXX. Discrete Sliding Mode Cintrok with Fuzzy Adaptive Reaching Law on 6-PEES Parallel Robot[C]. Intelligent System Design and Applications, Jinan, 2006: 649-652.(EI~收录号:20073210746529)
\end{publist}

\noindent\textbf{(二)申请及已获得的专利(无专利时此项不必列出)}
\begin{publist}
\item XXX,XXX. 一种温热外敷药制备方案:中国,88105607.3[P]. 1989-07-26.
\end{publist}

\noindent\textbf{(三)参与的科研项目及获奖情况}
\begin{publist}
\item	XXX,XXX. XX~气体静压轴承技术研究, XX~省自然科学基金项目.课题编号:XXXX.
\item XXX,XXX. XX~静载下预应力混凝土房屋结构设计统一理论. 黑江省科学技术二等奖, 2007.
\end{publist}
%\vfill
%\hangafter=1\hangindent=2em\noindent
%\setlength{\parindent}{2em}
\end{doctorpublication}
    % 所发文章
% % !Mode:: "TeX:UTF-8" 
\begin{publication}
\noindent\textbf{发表的相关论文}
\begin{publist}
\item	XXX,XXX. Static Oxidation Model of Al-Mg/C Dissipation Thermal Protection Materials[J]. Rare Metal Materials and Engineering, 2010, 39(Suppl. 1): 520-524.(SCI~收录,IDS号为~669JS,IF=0.16)
\item XXX,XXX. 精密超声振动切削单晶铜的计算机仿真研究[J]. 系统仿真学报,2007,19(4):738-741,753.(EI~收录号:20071310514841)
\item XXX,XXX. 局部多孔质气体静压轴向轴承静态特性的数值求解[J]. 摩擦学学报,2007(1):68-72.(EI~收录号:20071510544816)
\item XXX,XXX. 硬脆光学晶体材料超精密切削理论研究综述[J]. 机械工程学报,2003,39(8):15-22.(EI~收录号:2004088028875)
\item XXX,XXX. 基于遗传算法的超精密切削加工表面粗糙度预测模型的参数辨识以及切削参数优化[J]. 机械工程学报,2005,41(11):158-162.(EI~收录号:2006039650087)
\item XXX,XXX. Discrete Sliding Mode Cintrok with Fuzzy Adaptive Reaching Law on 6-PEES Parallel Robot[C]. Intelligent System Design and Applications, Jinan, 2006: 649-652.(EI~收录号:20073210746529)
\end{publist}

\noindent\textbf{(二)申请及已获得的专利(无专利时此项不必列出)}
\begin{publist}
\item XXX,XXX. 一种温热外敷药制备方案:中国,88105607.3[P]. 1989-07-26.
\end{publist}

\noindent\textbf{(三)参与的科研项目及获奖情况}
\begin{publist}
\item	XXX,XXX. XX~气体静压轴承技术研究, XX~省自然科学基金项目.课题编号:XXXX.
\item XXX,XXX. XX~静载下预应力混凝土房屋结构设计统一理论. 黑江省科学技术二等奖, 2007.
\end{publist}
%\vfill
%\hangafter=1\hangindent=2em\noindent
%\setlength{\parindent}{2em}
\end{publication}
    % 所发文章
% \include{back/resume}          % 博士学位论文有个人简介
% % !Mode:: "TeX:UTF-8"
\begin{correspondingaddr}
  \heiti\xiaosi
  \noindent 永久通讯地址: \par
  \noindent email: \par
  \noindent 电话: \par
\end{correspondingaddr}
 %通信地址
%%%%%%%%%%%%%%%%%%%%%%%%%%%%%%%%%%%%%%%%%%%%%%%%%%%%%%%%%%%%%%%%%%%%%%%%%%%%%%%% 
\end{document}
% Local Variables:
% TeX-engine: xetex
% End:
